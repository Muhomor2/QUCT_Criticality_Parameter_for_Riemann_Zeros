\documentclass[12pt]{article}
\usepackage{amsmath}
\usepackage{hyperref}
\usepackage{graphicx}
\usepackage{geometry}
\geometry{margin=1in}

\title{QUCT: A Simple Explanation of the Universal Criticality Parameter $\gamma^*$}
\author{Igor Chechelnitsky}
\date{2025}

\begin{document}

\maketitle

\begin{abstract}
This document explains, in simple and intuitive language, what the QUCT project is, what the criticality parameter $\gamma^*$ means, how it is computed, and why this parameter is important for the study of Riemann zeros and Random Matrix Theory. No advanced mathematical background is required.
\end{abstract}

\section{Introduction}
The QUCT project (Qadmon Universal Criticality Theory) investigates a simple question:

\textbf{Why do the zeros of the Riemann zeta function follow the same statistics as random matrices?}

This is one of the most famous observations in modern mathematics. Many people know \emph{that} it happens, but almost no one can explain \emph{why}.

Our approach introduces a new mathematical quantity, called the \textbf{criticality parameter} $\gamma^*$.

\section{What is $\gamma^*$?}
Imagine you have a long list of numbers (like the zeros of the zeta function). Between each pair there is a spacing. These spacings behave in a very special way: the small spacings appear less often than random, and the medium spacings appear more often. This is the ``repulsion'' typical of quantum chaotic systems.

The constant $\gamma^*$ is a number that describes exactly where the behaviour of spacings changes from one regime to another.

In QUCT, $\gamma^*$ is defined as the solution of a simple equation:
\[
F'''(\gamma^*) = 0.
\]

This means that $\gamma^*$ is the point where the curvature of a special function $F(\gamma)$ changes sign. This corresponds to a ``phase transition’’ in the structure of spacings.

\section{How do we compute $\gamma^*$?}
We use a function of the form:
\[
F'''(\gamma) = -A a^3 e^{-a\gamma} - B b^3 e^{-b\gamma} + 2\mu.
\]

The parameters $A, B, a, b, \mu$ come from the QUCT model.

We then use a numerical solver to find the value of $\gamma$ where $F'''(\gamma) = 0$.

The result is:
\[
\gamma^* \approx 0.3958242245.
\]

This value is extremely stable and appears in many independent computations.

\section{Why is $\gamma^*$ important?}
It is important because:
\begin{itemize}
    \item it matches the numerical behaviour of Riemann zeros,
    \item it predicts the transition point in the spacing distribution,
    \item it appears to be universal (independent of details),
    \item it gives a possible explanation why Riemann zeros behave like eigenvalues of random matrices (GUE).
\end{itemize}

\section{How to reproduce the result}
The repository contains the Python script:
\begin{itemize}
    \item \texttt{src/quct\_gamma\_root.py}
\end{itemize}

It computes $\gamma^*$ using high precision arithmetic.

Anyone can reproduce the same value with the same code.

\section{Summary}
This document gives an intuitive introduction to the meaning of the QUCT criticality parameter $\gamma^*$. For full mathematical details, please see the main technical report in the repository.
\end{document}
